%**************************************************************
% file contenente le impostazioni della tesi
%**************************************************************

%**************************************************************
% Frontespizio
%**************************************************************
\newcommand{\myName}{Marco Zanella}                             % autore
\newcommand{\myTitle}{Bioinformatiocs project}
\newcommand{\myUni}{University of Padua}           % università
\newcommand{\myFaculty}{Master's Degree in Computer Science}         % facoltà
\newcommand{\myDepartment}{Mathematics Department}          % dipartimento
\newcommand{\myLocation}{Padua}                                % dove
\newcommand{\myAA}{2017-2018}                                   % anno
\newcommand{\myCopyright}{CC-BY-SA-4.0}                         % copyright
\newcommand{\myVersion}{v0.0.1}                                 % versione
\newcommand{\myRelease}{
XXXXXXXXXXX}
\newcommand{\myIssue}{
XXXXXXXXXXX}
\newcommand{\myPullRequest}{
XXXXXXXXXXX}

%**************************************************************
% Short cuts
%**************************************************************
\newcommand{\newglsacr}[2]{\newacronym[see={[Glossary:]{#1}}]{a-#1}{#1}{#2}}

%**************************************************************
% Impostazioni di impaginazione
% see: http://wwwcdf.pd.infn.it/AppuntiLinux/a2547.htm
%**************************************************************

\setlength{\parindent}{14pt}   % larghezza rientro della prima riga
\setlength{\parskip}{0pt}   % distanza tra i paragrafi


%**************************************************************
% Impostazioni di caption
%**************************************************************
\captionsetup{
    tableposition=top,
    figureposition=bottom,
    font=small,
    format=hang,
    labelfont=bf
}

%**************************************************************
% Impostazioni di glossaries
%**************************************************************
\makeglossaries
\input{res/glossary} % database di termini

%**************************************************************
% Impostazioni di graphicx
%**************************************************************
\graphicspath{{res/img/}} % cartella dove sono riposte le immagini


%**************************************************************
% Impostazioni di hyperref
%**************************************************************
\hypersetup{
    %hyperfootnotes=false,
    %pdfpagelabels,
    %draft,	% = elimina tutti i link (utile per stampe in bianco e nero)
    colorlinks=true,
    linktocpage=true,
    pdfstartpage=1,
    pdfstartview=FitV,
    % decommenta la riga seguente per avere link in nero (per esempio per la
%stampa in bianco e nero)
    %colorlinks=false, linktocpage=false, pdfborder={0 0 0}, pdfstartpage=1,
%pdfstartview=FitV,
    breaklinks=true,
    pdfpagemode=UseNone,
    pageanchor=true,
    pdfpagemode=UseOutlines,
    plainpages=false,
    bookmarksnumbered,
    bookmarksopen=true,
    bookmarksopenlevel=1,
    hypertexnames=true,
    pdfhighlight=/O,
    %nesting=true,
    %frenchlinks,
    urlcolor=webbrown,
    linkcolor=webbrown,
    citecolor=webgreen,
    %pagecolor=RoyalBlue,
    %urlcolor=Black, linkcolor=Black, citecolor=Black, %pagecolor=Black,
    pdftitle={\myTitle},
    pdfauthor={\textcopyright\ \myName, \myUni, \myFaculty},
    pdfsubject={},
    pdfkeywords={},
    pdfcreator={pdfLaTeX},
    pdfproducer={LaTeX}
}

%**************************************************************
% Impostazioni di itemize
%**************************************************************
%\renewcommand{\labelitemi}{$\ast$}

%\renewcommand{\labelitemi}{$\bullet$}
%\renewcommand{\labelitemii}{$\cdot$}
%\renewcommand{\labelitemiii}{$\diamond$}
%\renewcommand{\labelitemiv}{$\ast$}


%**************************************************************
% Impostazioni di listings
%**************************************************************
\lstset{ 
  language=R,                     % the language of the code
  basicstyle=\tiny\ttfamily, % the size of the fonts that are used for the code
  numbers=left,                   % where to put the line-numbers
  numberstyle=\tiny\color{Blue},  % the style that is used for the line-numbers
  stepnumber=1,                   % the step between two line-numbers. If it is 1, each line
                                  % will be numbered
  numbersep=5pt,                  % how far the line-numbers are from the code
  backgroundcolor=\color{white},  % choose the background color. You must add \usepackage{color}
  showspaces=false,               % show spaces adding particular underscores
  showstringspaces=false,         % underline spaces within strings
  showtabs=false,                 % show tabs within strings adding particular underscores
  frame=single,                   % adds a frame around the code
  rulecolor=\color{black},        % if not set, the frame-color may be changed on line-breaks within not-black text (e.g. commens (green here))
  tabsize=2,                      % sets default tabsize to 2 spaces
  captionpos=b,                   % sets the caption-position to bottom
  breaklines=true,                % sets automatic line breaking
  breakatwhitespace=false,        % sets if automatic breaks should only happen at whitespace
  keywordstyle=\color{RoyalBlue},      % keyword style
  commentstyle=\color{Gray},   % comment style
  stringstyle=\color{ForestGreen}      % string literal style
} 


%**************************************************************
% Impostazioni di xcolor
%**************************************************************
\definecolor{webgreen}{rgb}{0,.5,0}
\definecolor{webbrown}{rgb}{.6,0,0}
\definecolor{Pantone}{RGB}{155,0,20}
\definecolor{GrigioLight}{RGB}{152, 152, 152}


%**************************************************************
% Altro
%**************************************************************

\newcommand{\omissis}{[\dots\negthinspace]} % produce [...]

% eccezioni all'algoritmo di sillabazione
\hyphenation
{
    ma-cro-istru-zio-ne
    gi-ral-din
}

\newcommand{\sectionname}{sezione}
%\addto\captionsitalian{\renewcommand{\figurename}{figura}
%                       \renewcommand{\tablename}{tabella}}

\newcommand{\glsfirstoccur}{\ap{{[g]}}}

\newcommand{\intro}[1]{\emph{\textsf{#1}}}

%-------------------INIZIO creazione subsubparagraph---------------------------
\makeatletter
\newcounter{subsubparagraph}[subparagraph]
\def\toclevel@subsubparagraph{6}
\renewcommand\thesubsubparagraph{%
  \thesubparagraph.\@arabic\c@subsubparagraph}
\newcommand\subsubparagraph{%
  \@startsection{subsubparagraph}    % counter
    {6}                              % level
    {\parindent}                     % indent
    {3.25ex \@plus 1ex \@minus .2ex} % beforeskip
    {-1em}                           % afterskip
    {\normalfont\normalsize\bfseries}}
\newcommand\l@subsubparagraph{\@dottedtocline{6}{13.5em}{5em}}
\newcommand{\subsubparagraphmark}[1]{}
\setcounter{tocdepth}{6}
\setcounter{secnumdepth}{6} % aggiunge contatore ai paragrafi
\makeatother
%-------------------FINE creazione subsubparagraph-----------------------------

%-------------------Capitoli personalizzati------------------------------------

\titleformat{\chapter}[display]
  {\normalsize \huge  \color{black}}%
  {\flushright\normalsize \color{Pantone}%
   \MakeUppercase{\chaptertitlename}\hspace{1ex}%
   {\fontsize{60}{60}\selectfont\thechapter}}%
  {10 pt}%
  {\bfseries\huge#1}%

%----------------FINE Capitoli personalizzati----------------------------------

%----------------INIZIO Parte personalizzata-----------------------------------

\renewcommand\thepart{\Alph{part}}

\newcommand\partnumfont{% font specification for the number
  \fontsize{304}{104}\color{white}\selectfont%
}

\newcommand\partnamefont{% font specification for the name "PART"
  \color{white}\huge\bfseries%
}

\titleformat{\part}[display]
   {\normalfont\huge\filleft}
   { }
   {20pt}
   {\begin{tikzpicture}[remember picture,overlay]
  \fill[GrigioLight]
    (current page.north west) rectangle ([yshift=-13cm]current page.north
east);
    \node[
      fill=Pantone,
      text width=2\paperwidth,
      rounded corners=6cm,
      text depth=12cm,
      anchor=center,
      inner sep=0pt] at ([yshift=21cm]current page.south west) (parttop)
    {\thepart};%
    \node[
      anchor=center,
      inner sep=0pt,
      outer sep=0pt] at ([xshift=16cm, yshift=6cm]parttop.south) (partnum)
    {\partnumfont\thepart};%
    \node[
      anchor=north east,
      align=right,
      inner sep=0pt] at ([yshift=10cm] current page.center)
    {\parbox{.7\textwidth}{\raggedleft\partnamefont\MakeUppercase{#1}}};
    \end{tikzpicture}%
}

%--------------- FINE parte personalizzata ------------------------------------
